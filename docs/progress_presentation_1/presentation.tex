\documentclass[10pt]{beamer}

\usetheme{metropolis}
\usepackage{appendixnumberbeamer}

\usepackage{booktabs}
\usepackage[scale=2]{ccicons}

\usepackage{pgfplots}
\usepgfplotslibrary{dateplot}

\usepackage{xspace}
\newcommand{\themename}{\textbf{\textsc{metropolis}}\xspace}

\title{Internship's progress}
\subtitle{First presentation}
\date{\today}
\author{Johyn Papin}
\institute{National Institute of Informatics}
% \titlegraphic{\hfill\includegraphics[height=1.5cm]{logo.pdf}}

\begin{document}

\maketitle

\begin{frame}{Table of contents}
  \setbeamertemplate{section in toc}[sections numbered]
  \tableofcontents[hideallsubsections]
\end{frame}

\section{Introduction}

\begin{frame}[fragile]{The project}
    The project concerns a Linked Data database containing recipes and ingredients.
    
    The project can be divided as follows:
    \begin{itemize}
        \item Extract prices and nutritional information from several sources
        \item Enrich the database with these data
        \item Use these data to calculate new information
    \end{itemize}
\end{frame}

\section{Progress}

\begin{frame}{What I've done}
    Here is what has been completed:
	\begin{itemize}
		\item Understand the linked data and its different formats
		\item Learn how to write SparQL queries
		\item Enrich the database with AGROVOC vocabulary
		\begin{itemize}
		    \item Read an n-triples file in GO
		    \item Represent an RDF graph in GO
		    \item Querying the AGROVOC API
		    \item Process n-triples as a stream
		\end{itemize}
		\item Extract prices and nutritional information from monoprix
		\begin{itemize}
		    \item Bypass the security of "monoprix".
		    \begin{itemize}
		        \item Use of TOR and request for a new circuit each time it is blocked
		        \item Use of a headless browser (firefox)
		    \end{itemize}
		    \item Scraping the menu, the categories and finally the products
		    \item Output the products as a stream
		    \item Scraping multiple pages at the same time (concurrently)
		\end{itemize}
	\end{itemize}
\end{frame}

\begin{frame}{What I'm doing}
    Here is what I'm doing now:
    \begin{itemize}
        \item Dockerize the program
        \item Use an indexing system to search monoprix data
        \item Use sparql to process the triples of the database in a stream fashion
        \item Use sparql to add triples to the database
    \end{itemize}
\end{frame}

\begin{frame}{Problem encountered}
    \begin{itemize}
        \item The security of monoprix. Solutions tested:
        \begin{itemize}
            \item simple GET queries
            \item as before but with cookies and random user-agents
            \item as before but with TOR proxy
            \item using a headless browser
            \item as before but with TOR proxy
            \item \textbf{as before but asking a new TOR circuit after each problem (and a new firefox profile)}
        \end{itemize}
        \item Race conditions and other concurrent issues. Solutions:
        \begin{itemize}
            \item Mutex (locks)
            \item Work queue
            \item State machine
        \end{itemize}
    \end{itemize}
\end{frame}

\section{Conclusion}

\begin{frame}{Summary}

  Get the source of this presentation on

  \begin{center}\url{github.com/johynpapin/yuuri}\end{center}

  The presentation \emph{itself} is licensed under a
  \href{http://creativecommons.org/licenses/by-sa/4.0/}{Creative Commons
  Attribution-ShareAlike 4.0 International License}.

  \begin{center}\ccbysa\end{center}

\end{frame}

\begin{frame}[standout]
  Questions?
\end{frame}

\appendix

\end{document}
