\documentclass[10pt]{beamer}

\usetheme{metropolis}
\usepackage{appendixnumberbeamer}

\usepackage{booktabs}
\usepackage[scale=2]{ccicons}

\usepackage{pgfplots}
\usepgfplotslibrary{dateplot}

\usepackage{xspace}
\newcommand{\themename}{\textbf{\textsc{metropolis}}\xspace}

\title{Internship's progress}
\subtitle{Second presentation}
\date{\today}
\author{Johyn Papin}
\institute{National Institute of Informatics}
% \titlegraphic{\hfill\includegraphics[height=1.5cm]{logo.pdf}}

\begin{document}

\maketitle

\begin{frame}{Table of contents}
  \setbeamertemplate{section in toc}[sections numbered]
  \tableofcontents[hideallsubsections]
\end{frame}

\section{Introduction}

\begin{frame}{Summary Diagram}
    \begin{tikzpicture}[remember picture,overlay]
        \node[at=(current page.center)] {
            \includegraphics[width=\paperwidth]{summary_diagram.png}
        };
    \end{tikzpicture}
\end{frame}

\section{Progress}

\begin{frame}{Since the last time}
    Here is what has been completed:
	\begin{itemize}
		\item Everything is on docker
	    \item ElasticSearch is used to index the monoprix data
	    \item Sparql is used to query the triples
	    \item These triples are enhanced by querying ElasticSearch
	\end{itemize}
\end{frame}

\begin{frame}{What I'm doing}
    Here is what I'm doing now:
    \begin{itemize}
        \item Be able to add the new triples on marmotta via sparql
        \item The web panel
        \begin{itemize}
            \item Must not slow down the rest
            \item Adding a middleware in the log system
            \item Opening a websocket server + a tiny restful API
            \item Using Vue.js:
            \begin{itemize}
                \item incrementally adoptable
                \item tiny
                \item M-V-VM
            \end{itemize}
        \end{itemize}
    \end{itemize}
\end{frame}

\section{Panel design}

\begin{frame}{Login Page}
    \includegraphics[width=11 cm]{login.png}
\end{frame}


\begin{frame}{Preparation Page}
    \includegraphics[width=11cm]{preparation.png}
\end{frame}

\section{Conclusion}

\begin{frame}{Summary}

  Get the source of this presentation on

  \begin{center}\url{github.com/johynpapin/yuuri}\end{center}

  The presentation \emph{itself} is licensed under a
  \href{http://creativecommons.org/licenses/by-sa/4.0/}{Creative Commons
  Attribution-ShareAlike 4.0 International License}.

  \begin{center}\ccbysa\end{center}

\end{frame}

\begin{frame}[standout]
  Questions?
\end{frame}

\appendix

\end{document}
